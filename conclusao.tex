% ---------------------------------------------------------------------------------------------------- %
%					ORIENTA��ES
% ---------------------------------------------------------------------------------------------------- %
% Conclus�o ou conclus�es
% � a s�ntese dos resultados do trabalho. Tem por finalidade recapitular sinteticamente os
% resultados da pesquisa elaborada.
% O autor manifestar� seu ponto de vista sobre os resultados obtidos, bem como sobre o
% seu alcance, sugerindo novas abordagens a serem consideradas em trabalhos seme-
% lhantes. Na conclus�o, o autor deve apresentar os resultados considerados mais impor-
% tantes e sua contribui��o ao tema.



\chapter{Conclus�o}
nao existe tecnica antiforense 100\% nem metodologia para analise infal�vel, nem 100\% segura e confi�vel. Foram mostrados diversas t�cnicas para atrasar e confundir o perito, e por vezes tornando o tempo de an�lise proibitiva. 

