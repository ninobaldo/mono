% ---------------------------------------------------------------------------------------------------- %
%					ORIENTA��ES
% ---------------------------------------------------------------------------------------------------- %
% Conclus�o ou conclus�es
% � a s�ntese dos resultados do trabalho. Tem por finalidade recapitular sinteticamente os
% resultados da pesquisa elaborada.
% O autor manifestar� seu ponto de vista sobre os resultados obtidos, bem como sobre o
% seu alcance, sugerindo novas abordagens a serem consideradas em trabalhos seme-
% lhantes. Na conclus�o, o autor deve apresentar os resultados considerados mais impor-
% tantes e sua contribui��o ao tema.
% ---------------------------------------------------------------------------------------------------- %

\chapter{Conclus�o}

O rootkit � um recurso sofisticado e valioso, pois seu desenvolvimento demanda tempo e profundo conhecimentos de arquitetura de computares e de sistemas operacionais. Muitas das suas t�cnicas podem ser usadas por outros tipos de \emph{malwares} e n�o necessariamente elas ser�o encontradas em todos os rootkits. Contudo dois aspectos s�o cruciais para definir um \emph{malwares} como rootkit que s�o conseguir acesso root no sistema operacional e manter-se escondido.

Durante as etapas do processo foram apresentados algumas t�cnicas antiforenses, mas como n�o existe t�cnica antiforense que seja indetect�vel e nem metodologia para an�lise infal�vel, segura e totalmente confi�vel.
Dessa forma as t�cnicas de antiforense assim como as de an�lise continuaram evolu�ndo e se confrontando numa eterna briga de gato e rato.

Foram mostrados diversas t�cnicas para atrasar e confundir o perito, e por vezes tornando o tempo de an�lise proibitiva.
Em futuros trabalhos poder�o ser abordados novas propostas e t�cnicas antiforense que poder�o ser aplicadas a rootkits como o uso de \emph{HPA} para persistir rootkits, esconder arquivos em dados em um \emph{file system} pr�prio ou ainda um \emph{packer} privado que decripta cada se��o com uma chave diferente ligada a harware do computador infectado. 
Tamb�m poder�o ser desenvolvidas novas t�cnicas para an�lise forense de rootkits como automatizadores que agilizam o processo.

Esse estudo reuniu conceitos de rootkits e antiforense para desenvolver um processo geral para an�lise \emph{post mortem} e seus empecilhos para ser usado em ambientes que podem ter sido subvertidos por rootkits.