\chapter*{Introdu��o}

O autor entende que o leitor ser� convencido a precave-se de supor que situa��es mais corriqueiras tenham s� e apenas a resposta mais obvia, desse modo evitando que sejam subestimadas. Al�m do conhecimento t�cnico e de foro que se espera de um perito, compreender que existem t�cnicas sofisticadas para destruir e/ou dificultar acesso a provas no ambito computacional � de fundamental necessidade para qualquer perito. A combina��o do uso de diversas t�cnicas avan�adas com maestria se mostra realmente desafiadora.

O objetivo desse texto n�o � de forma alguma desencorajar peritos em sua nobre luta di�ria superestimando todas as suas pr�ximas empreitadas t�o ut�is para a sociedade. Definitivamente esse n�o � o caso e sim mostrar que o perito pode se deparar com algum artefato que tenham sido usado por algu�m com o mesmo ou um maior conhecimento sobre o corpo de delito.

� claro que essa monografia apresenta apenas uma limitada incurs�o na antiforense focada na utiliza��o de rootkits ciente que ningu�m abordaria de forma completa tais assuntos. Portando em assuntos que fojem da mediocridade cotidiana e adentram nesse mundo instigante e revelador da an�lise de artefatos �nicos, se faz necess�rio que quanto maior o desafio maior dever� ser a dedica��o e paix�o do perito.




 Durante o avan�o do texto, no entanto, poder� se notar diversas formas de detec��o de rootkits al�m de indicios de seu uso em analises.