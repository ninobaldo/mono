\chapter*{Introdu��o}

O perito deve precaver-se de supor que situa��es mais corriqueiras tenham s� e apenas a resposta mais obvia desse modo evitando que sejam subestimadas. Al�m do conhecimento t�cnico e de foro que se espera de um perito, compreender que existem t�cnicas sofisticadas para destruir e/ou dificultar acesso a provas no ambito computacional � de fundamental necessidade para qualquer perito. A combina��o do uso de diversas t�cnicas avan�adas com maestria se mostra realmente desafiadora, pois quando isso ocorre o artefato investigado pode estar preparado para ser investigado removendo rastros e evitando a cri��o de evid�ncias.

Peritos em sua nobre luta di�ria podem se deparar, em uma das suas pr�ximas empreitadas t�o ut�is para a sociedade, com algum artefato que tenham sido subvertido por algu�m com o mesmo ou um maior conhecimento do que o seu sobre o corpo de delito periciado.

A monografia apresenta uma incurs�o na antiforense focada na utiliza��o de rootkits e durante o avan�o do trabalho, poder�o ser notadas diversas formas de detec��o de rootkits al�m de ind�cios de seu uso em an�lises.
%Portando em assuntos que fojem da mediocridade cotidiana e adentram nesse mundo instigante e revelador da an�lise de artefatos �nicos, se faz necess�rio que quanto maior o desafio, maior dever� ser a dedica��o e paix�o do perito.