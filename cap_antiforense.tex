% virou section no capitulo de an�lise

\chapter{Antiforense}

\citeonline{RH2006} define antiforense como m�todo para prevenir ou agir contra a ci�ncia usada a favor das leis civis e criminais que s�o aplicadas por �rg�os como a pol�cia. \citeonline{BERI2007}, amplia essa ideia mostrando que � mais que uma t�cnica usada, � uma abordagem crimosa. 
Assim todas as tentativas de interferir na exist�ncia, quantidade e/ou qualidade da evid�ncia de uma cena de crime, ou mesmo fazer a an�lise e exame das provas dif�ceis ou imposs�veis de realizar, para \citeonline{ROG2005} ser� considerada antiforense. 
Dessa forma conclu�mos resumidamente que a antiforense computacional pode ser definida como qualquer a��o praticada para obstruir, dificultar ou destruir evid�ncias ou provas no �mbito computacional.

Dentre as modalidades de antiforense computacional, \citeonline{BH2009} as categoriza em cinco grandes grupos, sendo eles: 

 \begin{table}[htb]
  \centering  
  \begin{tabular}{|l|l|}
  \hline
	\textbf{Tipo}		& \textbf{Descri��o}  \\ \hline
	Destrui��o de dados	& Destruir arquivos ou metadados \\ \hline
	Oculta��o de dados	& Salvar arquivos em lugares incomuns ou n�o convencionais \\ \hline
	Corrup��o de dados	& Compacta��o, criptografia ou mudar o seu \emph{file format} \\ \hline
	Fabrica��o de dados	& Introduzir \emph{known files} ou pistas falsas \\ \hline
	N�o uso de disco	& Contracep��o de dados ou inje��o em mem�ria \\ \hline
  \end{tabular}
  \caption{Tipos de antiforense \cite{BH2009}.}
  \label{tab:upx-comp}
\end{table}

Esses tipos ser�o apresentados no �ltimo cap�tulo durante as fases da an�lise \emph{post mortem}.




%Al�m de enumerar as modalidades antiforenses, voc� vai defini-las e explic�-las?  http://cyberforensics.purdue.edu/documents/AntiForensics_LockheedMartin09152005.pdf#documents/AntiForensics_LockheedMartin09152005.pdf#