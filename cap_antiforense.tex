\chapter{Antiforense Computacional}

\citeonline{RH2006} define antiforense como m�todo para prevenir ou agir contra a ci�ncia usada a favor das leis civis e criminais que s�o aplicadas pelos �rg�os como a pol�cia. J� \citeonline{BERI2007}, amplia essa ideia mostrando que � mais que uma t�cnica usada, � uma abordagem crimosa. Logo antiforense computacional pode ser definida como qualquer a��o praticada para obstruir, dificultar ou destruir evid�ncias ou provas no �mbito computacional.

Dentre as modalidades de antiforense computacional, \citeonline{BH2009} as categoriza em cinco grandes grupos, sendo eles: destrui��o de dados, oculta��o de dados, corrup��o de dados, fabrica��o de dados e elimina��o da fonte de dados.

