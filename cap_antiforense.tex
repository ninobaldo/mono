\chapter{Antiforense}

\citeonline{RH2006} define antiforense como m�todo para prevenir ou agir contra a ci�ncia usada a favor das leis civis e criminais que s�o aplicadas por �rg�os como a pol�cia. \citeonline{BERI2007}, amplia essa ideia mostrando que � mais que uma t�cnica usada, � uma abordagem crimosa. Assim todas as tentativas de afetar negativamente a exist�ncia, quantidade e / ou qualidade da evid�ncia de uma cena de crime, ou fazer a an�lise e exame das provas dif�ceis ou imposs�veis de realizar, para \citeonline{ROG2005} ser� considerada antiforense. Dessa forma conclu�mos resumidamente que a antiforense computacional pode ser definida como qualquer a��o praticada para obstruir, dificultar ou destruir evid�ncias ou provas no �mbito computacional.

Dentre as modalidades de antiforense computacional, \citeonline{BH2009} as categoriza em cinco grandes grupos, sendo eles: destrui��o de dados, oculta��o de dados, corrup��o de dados, fabrica��o de dados e elimina��o da fonte de dados.

%Al�m de enumerar as modalidades antiforenses, voc� vai defini-las e explic�-las?  http://cyberforensics.purdue.edu/documents/AntiForensics_LockheedMartin09152005.pdf#documents/AntiForensics_LockheedMartin09152005.pdf#