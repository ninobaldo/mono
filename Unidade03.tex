%\part{Timeline - Desenvolvimento da Monografia}

%\chapter{Desenvolvimento da Monografia}

\chapter{Unidade 03 - 7 setembro - 13 setembro}

\section{Proposto no tópico}

- Objetivos;
- Justificativa/relevância da pesquisa;
- Fundamentação Teórica.


\subsection{Resolução}


\begin{itemize}
  \item 
\end{itemize}

\section{Proposto no forum}

1. Defina o seu tema de pesquisa e apresente 03(três) referências bibliográficas que servirão de base para o desenvolvimento da sua pesquisa.
2. A partir de seu tema, desenvolva o ‘problema de pesquisa’
3. Desenvolva a estrutura de seu projeto de pesquisa, estabelecendo os tópicos e descrevendo, sumariamente, o que será desenvolvido dentro de cada tópico.

\subsection{Resolução}

tema: Antiforense com rootkits em ambientes windows 7 referencias: Rootkit Arsenal, Windows Internal ed 6 Part 1 e Sistemas operacionais modernos problematica: apresentar o uso de rootkits para anti-forense e maneiras de identificar e/ou anular o efeito do mesmo.

\begin{itemize}
	\item 
\end{itemize}

\section{Resposta da orientadora}
Monday, 17 September 2012
13:54: Anti-forense com rootkits em ambientes windows 7
por ALVARO VILOBALDO RIOS DA SILVA . - quarta, 12 setembro 2012, 20:35

Alvaro,
Para a entrega desta semana, releia sua justificativa apresentada na semana letiva anterior e reorganize os itens. Coloque em caixa alta, para destaque, algumas orientações:
ACRESCENTE O PROBLEMA DE PESQUISA, O QUE VOCÊ COLOCOU EM “PROBLEMÁTICA”, É A QUESTÃO (COMO APRESENTAR....?) E OBJETIVO ...
RELEIA OS SLIDES DAS AULAS ANTERIORES, E EM CASO DE DÚVIDAS, MANDE E-MAIL.
tema: Anti-forense com rootkits em ambientes windows 7
referencias: Rootkit Arsenal, Windows Internal ed 6 Part 1 e Sistemas operacionais modernos problematica: COMO apresentar o uso de rootkits para anti-forense (É SEM HÍFEN) e maneiras de identificar e/ou anular o efeito do mesmo??? (substitua “o mesmo” por “ele)

À DISPOSIÇÃO,
PROFESSORA IVETE IRENE 

