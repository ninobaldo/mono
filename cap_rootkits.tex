\chapter{Rootkits}

% Expor o que � um rootkit;
% Como o rootkit pode impedir a forma��o de provas;
% Mostrar o que ele pode fazer usando se poss�vel com exemplos reais;

\begin{citacao}
``Rootkit � uma cole��o de ferramentas (por exemplo, bin�rios, scripts, arquivos de configura��o) que permitem que os invasores para ocultar sua atividade em um computador, de modo que eles podem secretamente monitorar e controlar o sistema por um per�odo prolongado.''\cite{BILL2009}.
\end{citacao}.

No universo \emph{UNIX\footnote{Um sistema operacional multiusu�rio amplamente utilizado.}} ou \emph{UNIX-like\footnote{Sistemas Operacionais baseados no Unix, como o GNU/Linux}} a conta de usu�rio com menor restri��o de seguran�a � refer�nciada como conta \emph{root} sendo que em alguns sistemas o nome de usu�rio � literalmente root, mas isso � apenas uma conven��o hist�rica do que uma imposi��o \cite{BILL2009}. Enquanto \emph{kit} significa conjunto de pe�as \cite{DIING}. 
