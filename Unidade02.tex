%\part{Timeline - Desenvolvimento da Monografia}

%\chapter{Desenvolvimento da Monografia}

\chapter{Unidade 02 - 31 agosto - 6 setembro}

\section{Proposto no tópico}

Tema;

Delimitação do Tema;

Problematização e hipóteses.

\subsection{Resolução}


\begin{itemize}
  \item Tema: Antiforense.
  \item Delimitação do Tema: Antiforense em Windows 7 com uso de rootkits.
  \item Problematização: Técnicas podem ser usadas para ocultar processos ou ações? 
  \item  Hipóteses: Rootkits e bombas lógicas são exemplos de como um usuário avançado pode disfarçar, dificultar ou impossíbilidar a ação de um perito forense na obtenção de provas.
\end{itemize}

\section{Proposto no forum}

-Tema: 

- delimitação (se já conseguiu):

- Justificativa pessoal para a escolha (se ainda não apresentou, apresentar)

- Relevância social  do tema: O que está acontecendo social/ cultural ou economicamente que justifica ou valida o tema? (PREPARAÇÃO PARA A JUSTIFICATIVA)

-Que problemas o estudo do tema pode resolver? (PREPARAÇÃO PAR AO PROBLEMA DE PESQUISA E OBJETIVO)

-Que repercussões e/ou contribuições o estudo do tema pode trazer no nível social, cultural e/ou econômico?(PREPARAÇÃO PARA A JUSTIFICATIVA E PROBLEMA DE PESQUISA)

- O que eu já sei sobre o assunto? (LEVANTAMENTO DA VIABILIDADE DO TEMA)

- Que trabalhos acadêmicos  sobre o tema já existem? (LEVANTAMENTO BIBLIOGRÁFICO/ ESTADO DA ARTE)

-Que trabalhos de temas afins já há sobre o tema? (LEVANTAMENTO BIBLIOGRÁFICO )

- Até o período da entrega da monografia o que será possível efetivamente fazer/ desenvolver em relação à pesquisa? (LEVANTAMENTO DA VIABILIDADE DO TEMA EM RELAÇÃO AO TEMPO )

\subsection{Resolução}


\begin{itemize}
	\item Tema: Antiforense

	\item delimitação: Antiforense em Windows 7 com uso de rootkits.

	\item Justificativa pessoal para a escolha: Acredito o perito deva esta preparado para a acao de um usuario avancado que conheca bem o OS atacado/comprometido/usado e que saiba bem o que esta fazendo. Mesmo que nao seja algo corriqueiro na rotina da grande maioria dos profissionais encontrar um atacante de alto nivel sera interessante escrever sobre como dificultar/impossibilitar o trabalho do perito e como o mesmo poderia evitar a armadilha de achar que no corpo (corpo de delito) investigado nao exista nada.

	\item Relevância social  do tema: O que está acontecendo social/ cultural ou economicamente que justifica ou valida o tema? (PREPARAÇÃO PARA A JUSTIFICATIVA): Por que invadir um país armado botando vidas de centenas em risco se você pode destruir centrífugas de enriquecimento de urânio de forma sigilosa do outro lado do planeta? Conforme o mundo se digitalizou se digitalizaram-se também as suas ameaças, onde antes se podia ver mesmo que por instantes misseis ou bombas sendo lançada hoje temos inumeras ameaças invísiveis que podem causar tanto estrago quanto, contudo pelo príncipio de Locard o problema dessas ameaças invísiveis é que elas podem não ser tão invisiveis assim, levando a procura de metodos antiforense mais eficazes.

	\item Que problemas o estudo do tema pode resolver? (PREPARAÇÃO PAR AO PROBLEMA DE PESQUISA E OBJETIVO): Pode esclarecer diversos tópicos obscuros a respeito do que pode ser encontrado em investigações forenses quando o perito se ve de frente com ameaças persistentes e elaboradas, ou seja, ajuda a resolver casos onde existiu antiforense.

	\item Que repercussões e/ou contribuições o estudo do tema pode trazer no nível social, cultural e/ou econômico?(PREPARAÇÃO PARA A JUSTIFICATIVA E PROBLEMA DE PESQUISA): Pode ajudar a traçar um processo bem elaborado de trabalho que seja rápida e eficiente sem deixar brechas que permitam ou ajudem ações antiforenses. Com um processo bem definido o perito tende a diminuir o tempo de análise e um melhor aproveitamento das mesmas. Provendo mais qualidade e mais precisam nas futuras análises.

	\item O que eu já sei sobre o assunto? (LEVANTAMENTO DA VIABILIDADE DO TEMA): A teoria de como rootkits funcionam e como capitura-los na ram.


	\item Que trabalhos acadêmicos  sobre o tema já existem? (LEVANTAMENTO BIBLIOGRÁFICO/ ESTADO DA ARTE)

	\item Que trabalhos de temas afins já há sobre o tema? (LEVANTAMENTO BIBLIOGRÁFICO )

	\item Até o período da entrega da monografia o que será possível efetivamente fazer/ desenvolver em relação à pesquisa? (LEVANTAMENTO DA VIABILIDADE DO TEMA EM RELAÇÃO AO TEMPO )

\end{itemize}

\section{Resposta da orientadora}
Tuesday, 11 September 2012
17:05: Álvaro,
Você já fez uma escolha, mas, na atividade dessa semana letiva, apresente as delimitações da pesquisa, reorganizando as informações já apresentadas e acrescentando e desenvolvendo as demais. São necessárias as delimitações, nessa etapa, para que comecemos a desenvolver a redação do projeto. Conte comigo! Seguem os itens:
Tema:
Delimitação do tema:
Justificativa
Problema de pesquisa (QUESTÃO QUE QUER RESPONDER)
Objetivos da pesquisa:
Bibliografia levantada:
