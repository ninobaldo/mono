\chapter*{Tema}

Antiforense em Windows 7 com uso de rootkits.

\section{Problematização e hipóteses}

Técnicas podem ser usadas para ocultar processos ou ações? Rootkits e bombas lógicas são exemplos de como um usuário avançado pode disfarçar, dificultar ou impossíbilidar a ação de um perito forense na obtenção de provas.

\subsection{Que problemas o estudo do tema pode resolver? (PREPARAÇÃO PAR AO PROBLEMA DE PESQUISA E OBJETIVO)}

Pode esclarecer diversos tópicos obscuros a respeito do que pode ser encontrado em investigações forenses quando o perito se ve de frente com ameaças persistentes e elaboradas, ou seja, ajuda a resolver casos onde existiu antiforense.

\subsection{Que repercussões e/ou contribuições o estudo do tema pode trazer no nível social, cultural e/ou econômico?(PREPARAÇÃO PARA A JUSTIFICATIVA E PROBLEMA DE PESQUISA)}

Pode ajudar a traçar um processo bem elaborado de trabalho que seja rápida e eficiente sem deixar brechas que permitam ou ajudem ações antiforenses. Com um processo bem definido o perito tende a diminuir o tempo de análise e um melhor aproveitamento das mesmas. Provendo mais qualidade e mais precisam nas futuras análises.

\section{Justificativa pessoal}
O perito deve estar preparado para a ação de um usuário avançado que conheça bem o sistema operacional atacado,comprometido ou usado. Mesmo que não seja algo corriqueiro na rotina da grande maioria dos profissionais, encontrar um atacante de alto nivel trará novos desafios e obstáculos tão poucos corriqueiros. Saber como dificultar ou impossibilitar o trabalho do perito é como ele poderá evitar a armadilha de achar que no corpo (corpo de delito) investigado não existe nada.

\section{Justificativa}

\subsection{Relevância social  do tema: O que está acontecendo social/ cultural ou economicamente que justifica ou valida o tema? (PREPARAÇÃO PARA A JUSTIFICATIVA)}

Por que invadir um país armado botando vidas de centenas em risco se você pode destruir centrífugas de enriquecimento de urânio de forma sigilosa do outro lado do planeta? Conforme o mundo se digitalizou se digitalizaram-se também as suas ameaças, onde antes se podia ver mesmo que por instantes misseis ou bombas sendo lançada hoje temos inumeras ameaças invísiveis que podem causar tanto estrago quanto, contudo pelo príncipio de Locard o problema dessas ameaças invísiveis é que elas podem não ser tão invisiveis assim, levando a procura de metodos antiforense mais eficazes.

------------------------------------------------------------
Sun Tzu (chinês simplificado: 孙武; chinês tradicional: 孫武; pinyin: Sūn Wǔ) (544 a.C. - 456 a.C.) há mais ou menos 500 anos de cristo já dizia que se conhecer a si mesmo e ao adversário não temerá o resultado de mil batalhas.

Conhecendo como essas técnicas podem ser usadas o períto pode se precaver e saber agir no momento de detectar o uso dos mesmos. O perito deve tomar alguns cuidados na análise de computadores, pois o mesmo pode ter sido usado por um usuário avançado o que muda razoavelmente a sua abordagem.
------------------------------------------------------------
