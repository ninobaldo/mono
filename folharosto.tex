% ---------------------------------------------------------------------------------------------------- %
%					ORIENTAÇÕES
% ---------------------------------------------------------------------------------------------------- %
% Folha de rosto
% Elemento obrigatório, contém dados essenciais à identificação do trabalho (ANEXOS AC
% - AF), os quais devem aparecer na seguinte ordem:
% ---------------------------------------------------------------------------------------------------- %
%				 Anverso da folha de rosto
% ---------------------------------------------------------------------------------------------------- %
% · nome do autor: responsável intelectual ou artístico do trabalho;
% · título principal do trabalho – claro, preciso, com palavras que identifiquem o seu conteú-
% do e possibilitem a indexação e recuperação da informação;
% · subtítulo (se houver) – deve ser evidenciada a sua subordinação ao título principal,
% precedido de dois-pontos;
% · número de volumes – se houver mais de um, deve constar em cada folha de rosto, a sua
% respectiva especificação;
% · finalidade do trabalho – devem constar: natureza do trabalho, curso, instituição e área de
% concentração;
% · nome do orientador e do co-orientador (se houver);
% · local (cidade) da instituição onde deve ser apresentado;
% · ano de depósito (da entrega).
% ---------------------------------------------------------------------------------------------------- %

\folhaderosto

% ---------------------------------------------------------------------------------------------------- %
% 					Verso da folha de rosto
% ---------------------------------------------------------------------------------------------------- %
% O verso da folha de rosto é o elemento obrigatório onde se encontra a ficha catalográfica
% do trabalho, segundo o Código de Catalogação Anglo-Americano vigente (ANEXOS AG -
% AH). Para elaboração dessa ficha, recomenda-se a solicitação dos serviços de um biblio-
% tecário.

% <LaTeX>
% \def\edicao#1{\gdef\@edicao{#1}} % edi��o
% \def\fichacat#1{\gdef\@fichacat{#1}} % texto da ficha
% catalogr�fica
% \edicao{}
% \fichacat{}
% 
% %% dados da ficha catalogr�fica
% % nome do curso
% \def\curso#1{\gdef\@curso{#1}}
% % primeiro autor
% \def\fcautor#1{\gdef\@fcautor{#1}}
% % autores
% \def\fcautores#1{\gdef\@fcautores{#1}}
% % primeiro autor
% \def\fccatalogacao#1{\gdef\@fccatalogacao{#1}}
% % classicacao 1
% \def\fcclasi#1{\gdef\@fcclasi{#1}}
% % classicacao 2
% \def\fcclasii#1{\gdef\@fcclasii{#1}}
% \fcclasii{}
% % classicacao 3
% \def\fcclasiii#1{\gdef\@fcclasiii{#1}}
% \fcclasiii{}
% 
% \newcommand{\tab}{\hspace*{0.7cm}}
% 
% \fichacat{% texto da ficha catalogr�fica
%   \@fcautor \\
%   \tab \@title / \@fcautores. 
%   -\,- \@edicao.ed. Lavras: UFLA/FAEPE, 2004.\\
%   \tab \pageref{LastPage} p. : il. - \@curso.\\ 
%   \\
%   \tab Bibliografia.\\
%   \\
%   \tab \@fccatalogacao\\
%   \\
%   \tab \hfill 
%   \begin{tabular}{r@{-}l} 
%   CDD & \@fcclasi 
%   \ifthenelse{\equal{\@fcclasii}{\@empty}}{}{
%   \\ & \@fcclasii 
%   }
%   \ifthenelse{\equal{\@fcclasiii}{\@empty}}{}{
%   \\ & \@fcclasiii 
%   }
%   \end{tabular}
%   \tab
% }
% 
% %%% trecho de codigo cortado,  a ficha era 
% %%% impressa junto com t�tulo, etc. etc. 
% 
%     \begin{center}
%       \bfseries
%       Ficha Catalogr�fica preparada pela Divis�o de
% Processos T�cnicos \\
%       da Biblioteca Central da UFLA
%     \end{center}
% 
%    \vspace{-0.3cm}
% 
%     \newlength{\mylen}
%     \addtolength{\mylen}{\textwidth}
%     \addtolength{\mylen}{-1cm}
%     \newlength{\mylenn}
%     \addtolength{\mylenn}{\textwidth}
%     \addtolength{\mylenn}{-2cm}
% 
%     \noindent \hfill  \fbox{\parbox[c]{\mylen}{
%         \vspace{0.5cm}
%         \hfill \parbox[c]{\mylenn}{
%           \@fichacat
%           } \hfill
%         \vspace{0.3cm}
%         }
%       } \hfill
% 
% </LaTeX>
