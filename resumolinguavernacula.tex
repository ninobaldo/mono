\chapter*{Resumo}

Esta monografia � um estudo sobre antiforense com utiliza��o de rootkits sendo desenvolvida como trabalho de conclus�o de curso \emph{Latu sensu} em computa��o forense, pela Universidade Presbiteriana Mackenzie de S�o Paulo, visando mostrar como funciona um rootkit e como ele pode ser usado na antiforense. Defini��es de antiforense e rootkits com suas pr�ticas comuns e respectivos hist�ricos junto com algumas t�cnicas de subvers�o demonstrar�o que o uso combinado de t�cnicas antiforenses incorporadas a rootkits cria mecanimos de destru���o de provas de forma sistem�tica e efici�nte. 




\section*{Apenas did�tico}

Professora Ivete, para fins did�ticos e de melhoria na nossa comunica��o eu deixo essa se��o. Lembrando que ela ser� removida antes da entrega final, servindo apenas para demonstrar minha linha de racion�o.

Solicitado pela senhora nas corre��es: ``Apresente, de forma resumida, o tema, os objetivos, a metodologia e as hip�teses.''


Tema: Esta monografia � um estudo sobre antiforense com utiliza��o de rootkits [...]

Objetivo: [...] mostrar como funciona um rootkit e como ele pode ser usado na antiforense.

Metodologia � apresentar: Defini��es de antiforense e rootkits com suas pr�ticas comuns e respectivos hist�ricos junto com algumas t�cnicas de subvers�o [...]

A Hip�tose � que: [...] o uso combinado de t�cnicas antiforenses incorporadas a rootkits cria mecanimos de destru���o de provas de forma sistem�tica e efici�nte.