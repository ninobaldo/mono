% ---------------------------------------------------------------------------------------------------- %
%					ORIENTA��ES
% ---------------------------------------------------------------------------------------------------- %
% Resumo na l�ngua vern�cula
% Elemento obrigat�rio, consiste na apresenta��o concisa dos pontos relevantes do texto.
% Elaborado em portugu�s, p�e em evid�ncia as mat�rias mais importantes do conte�do,
% visando a fornecer, dessa forma, meios para a decis�o do leitor sobre a conveni�ncia, ou
% n�o, de consultar o texto completo.
% Redigido pelo pr�prio autor, contendo de 150 a 500 palavras e deve dar uma vis�o conci-
% sa e clara do conte�do, ou seja, as id�ias principais do texto e as conclus�es do trabalho.
% Na apresenta��o, o resumo deve ser redigido em par�grafo �nico, utilizando-se espa�o
% de 1,5 cm, com frases claras e concatenadas e seguido das palavras mais representati-
% vas do conte�do do trabalho, isto �, palavras-chave e/ou descritores (ANEXO AQ).



\begin{resumo}
Esta monografia � um estudo sobre rootkits e antiforense com a vis�o de um perito computacional forense.
Baseado no funcionamento e premissas do rootkit ser� apresentado um processo para an�lise \emph{post mortem} e as t�cnicas antiforenses em cada uma das suas etapas.
Defini��es de antiforense e rootkits com suas pr�ticas comuns e respectivos hist�ricos junto com algumas t�cnicas de subvers�o demonstrar�o que o uso combinado de t�cnicas antiforenses incorporadas a rootkits cria mecanimos de destru���o de provas de forma sistem�tica e efici�nte.

\textbf{Palavras-chave:} antiforense, rootkit, forense, provas, evid�ncias.
\end{resumo}